\documentclass[journal,12pt,twocolumn]{IEEEtran}

\usepackage{setspace}
\usepackage{gensymb}

\singlespacing


\usepackage[cmex10]{amsmath}

\usepackage{amsthm}

\usepackage{mathrsfs}
\usepackage{txfonts}
\usepackage{stfloats}
\usepackage{bm}
\usepackage{cite}
\usepackage{cases}
\usepackage{subfig}

\usepackage{longtable}
\usepackage{multirow}

\usepackage{enumitem}
\usepackage{mathtools}
\usepackage{steinmetz}
\usepackage{tikz}
\usepackage{circuitikz}
\usepackage{verbatim}
\usepackage{tfrupee}
\usepackage[breaklinks=true]{hyperref}
\usepackage{graphicx}
\usepackage{tkz-euclide}
\usepackage{float}

\usetikzlibrary{calc,math}
\usepackage{listings}
    \usepackage{color}                                            %%
    \usepackage{array}                                            %%
    \usepackage{longtable}                                        %%
    \usepackage{calc}                                             %%
    \usepackage{multirow}                                         %%
    \usepackage{hhline}                                           %%
    \usepackage{ifthen}                                           %%
    \usepackage{lscape}     
\usepackage{multicol}
\usepackage{chngcntr}

\DeclareMathOperator*{\Res}{Res}

\renewcommand\thesection{\arabic{section}}
\renewcommand\thesubsection{\thesection.\arabic{subsection}}
\renewcommand\thesubsubsection{\thesubsection.\arabic{subsubsection}}

\renewcommand\thesectiondis{\arabic{section}}
\renewcommand\thesubsectiondis{\thesectiondis.\arabic{subsection}}
\renewcommand\thesubsubsectiondis{\thesubsectiondis.\arabic{subsubsection}}


\hyphenation{op-tical net-works semi-conduc-tor}
\def\inputGnumericTable{}                                 %%

\lstset{
%language=C,
frame=single, 
breaklines=true,
columns=fullflexible
}
\begin{document}


\newtheorem{theorem}{Theorem}[section]
\newtheorem{problem}{Problem}
\newtheorem{proposition}{Proposition}[section]
\newtheorem{lemma}{Lemma}[section]
\newtheorem{corollary}[theorem]{Corollary}
\newtheorem{example}{Example}[section]
\newtheorem{definition}[problem]{Definition}

\newcommand{\BEQA}{\begin{eqnarray}}
\newcommand{\EEQA}{\end{eqnarray}}
\newcommand{\define}{\stackrel{\triangle}{=}}
\bibliographystyle{IEEEtran}
\providecommand{\mbf}{\mathbf}
\providecommand{\pr}[1]{\ensuremath{\Pr\left(#1\right)}}
\providecommand{\qfunc}[1]{\ensuremath{Q\left(#1\right)}}
\providecommand{\sbrak}[1]{\ensuremath{{}\left[#1\right]}}
\providecommand{\lsbrak}[1]{\ensuremath{{}\left[#1\right.}}
\providecommand{\rsbrak}[1]{\ensuremath{{}\left.#1\right]}}
\providecommand{\brak}[1]{\ensuremath{\left(#1\right)}}
\providecommand{\lbrak}[1]{\ensuremath{\left(#1\right.}}
\providecommand{\rbrak}[1]{\ensuremath{\left.#1\right)}}
\providecommand{\cbrak}[1]{\ensuremath{\left\{#1\right\}}}
\providecommand{\lcbrak}[1]{\ensuremath{\left\{#1\right.}}
\providecommand{\rcbrak}[1]{\ensuremath{\left.#1\right\}}}
\theoremstyle{remark}
\newtheorem{rem}{Remark}
\newcommand{\sgn}{\mathop{\mathrm{sgn}}}
\providecommand{\abs}[1]{\lvert#1\vert}
\providecommand{\res}[1]{\Res\displaylimits_{#1}} 
\providecommand{\norm}[1]{\lVert#1\rVert}
%\providecommand{\norm}[1]{\lVert#1\rVert}
\providecommand{\mtx}[1]{\mathbf{#1}}
\providecommand{\mean}[1]{E[ #1 ]}
\providecommand{\fourier}{\overset{\mathcal{F}}{ \rightleftharpoons}}
%\providecommand{\hilbert}{\overset{\mathcal{H}}{ \rightleftharpoons}}
\providecommand{\system}{\overset{\mathcal{H}}{ \longleftrightarrow}}
	%\newcommand{\solution}[2]{\textbf{Solution:}{#1}}
\newcommand{\solution}{\noindent \textbf{Solution: }}
\newcommand{\cosec}{\,\text{cosec}\,}
\providecommand{\dec}[2]{\ensuremath{\overset{#1}{\underset{#2}{\gtrless}}}}
\newcommand{\myvec}[1]{\ensuremath{\begin{pmatrix}#1\end{pmatrix}}}
\newcommand{\mydet}[1]{\ensuremath{\begin{vmatrix}#1\end{vmatrix}}}
\numberwithin{equation}{subsection}
\makeatletter
\@addtoreset{figure}{problem}
\makeatother
\let\StandardTheFigure\thefigure
\let\vec\mathbf
\renewcommand{\thefigure}{\theproblem}
\def\putbox#1#2#3{\makebox[0in][l]{\makebox[#1][l]{}\raisebox{\baselineskip}[0in][0in]{\raisebox{#2}[0in][0in]{#3}}}}
     \def\rightbox#1{\makebox[0in][r]{#1}}
     \def\centbox#1{\makebox[0in]{#1}}
     \def\topbox#1{\raisebox{-\baselineskip}[0in][0in]{#1}}
     \def\midbox#1{\raisebox{-0.5\baselineskip}[0in][0in]{#1}}
\vspace{3cm}
\title{ASSIGNMENT-11}
\author{Unnati Gupta}
\maketitle
\newpage
\bigskip
\renewcommand{\thefigure}{\theenumi}
\renewcommand{\thetable}{\theenumi}
Download all python codes from 
\begin{lstlisting}
https://github.com/unnatigupta2320/Assignment_11/blob/master/codes.py
\end{lstlisting}
%
and latex-tikz codes from 
%
\begin{lstlisting}
https://github.com/unnatigupta2320/Assignment_11
\end{lstlisting}
%
\section{Question No. 2.30}
A fruit grower can use two types of fertilizer in his garden, brand P and brand Q.The amounts(in kg) of nitrogen, phosphoric acid,potash and chlorine in a bag of each brand are given in the table.Tests indicate that garden needs atleast 240 kg of phosphoric acid,atleast 270 kg of potash and atmost 310 kg of chlorine. If the grower wants to minimise the amount of nitrogen added to garden, how many bags of each brand should be used?What is the minimum amount of nitrogen added in the ground?
\numberwithin{table}{section}
\begin{table}[!ht]
\centering
\resizebox{\columnwidth}{!}{\begin{tabular}{|c|c|c|} 
\hline
 & \textbf{Brand P }& \textbf{Brand Q }\\
\hline
Nitrogen & 3  & 3.5  \\ 
\hline
Phosphoric Acid & 1 & 2 \\ 
\hline
Potash & 3 & 1.5\\ 
\hline
Chlorine & 1.5 & 2 \\ 
\hline
\end{tabular}}
\caption{kg per bag}
\label{tab:table1}
\end{table}
\section{Solution}
\begin{itemize}
\item All the data can be tabularised as:
\numberwithin{table}{section}
\begin{table}[!ht]
\centering
\resizebox{\columnwidth}{!}{\begin{tabular}{|c|c|c|c|} 
\hline
 & \textbf{Brand P }& \textbf{Brand Q } &Amounts Required\\
\hline
Nitrogen & 3  & 3.5 & $?$ \\ 
\hline
Phosphoric Acid & 1 & 2 &$\geq 240$ kg \\ 
\hline
Potash & 3 & 1.5 &$\geq 270$ kg\\ 
\hline
Chlorine & 1.5 & 2 &$\leq 310$ kg\\ 
\hline
\end{tabular}}
\caption{Requirements of fertilizers}
\label{tab:table2}
\end{table}
\item Let the number of bags of Brand P be $x$ $\And$
\item The number of bags of Brand Q be $y$ such that : 
\begin{align}
    x \geq 0 
    \\
    y \geq 0 
\end{align}
\item From the data given we have:
\begin{align}
    x+2y &\geq 240 \\
   \implies   -x-2y &\leq -240
\end{align}
and,
\begin{align}
    3x+1.5y &\geq 270 \\
    \implies -x-0.5y &\leq -90
\end{align}
and,
\begin{align}
     1.5x+2y &\leq 310 \\
\end{align}
$\therefore$ The minimizing function is:
\begin{align}
        \min_{\vec{x}} Z &= \myvec{3& 3.5}\vec{x}\\
        s.t. \quad 
        \myvec{-1 & -2\\ -1 & -0.5 \\ 1.5 & 2 }\vec{x} &\preceq \myvec{-240\\-90\\310} \\
        \vec{-x} &\preceq \vec{0}
\end{align}
\item The Lagrangian function can be given as:
\begin{equation}
\begin{aligned}
    &L(\vec{x},\boldsymbol{\lambda}) \\ &= \myvec{3 & 3.5}\vec{x}+\lcbrak{\sbrak{\myvec{-1 & -2}\vec{x}+240}} \\ &+ \sbrak{\myvec{-1 & -0.5}\vec{x}+90} +\sbrak{\myvec{1.5 & 2}\vec{x}-310} \\ &+ \sbrak{\myvec{-1 & 0}\vec{x}} +\rcbrak{\sbrak{\myvec{0 & -1}\vec{x}}}\boldsymbol{\lambda}
\end{aligned}
\end{equation}
where,
\begin{align}
    \boldsymbol{\lambda} &= \myvec{\lambda_1 \\ \lambda_2 \\ \lambda_3 \\ \lambda_4 \\ \lambda_5 \\ \lambda_6}
\end{align}
\item Now, we have
\begin{align}
    \nabla L(\vec{x},\boldsymbol{\lambda}) &= \myvec{3+ \myvec{-1 & -1 & 1.5 & -1 & 0 }\boldsymbol{\lambda}\\ 3.5+\myvec{-2 & -0.5 & 2 & 0 & -1}\boldsymbol{\lambda} \\ \myvec{-1 & -2}\vec{x}+240 \\ \myvec{-1 & -0.5}\vec{x}+90 \\ \myvec{1.5 & 2}\vec{x}-310 \\ \myvec{-1 & 0}\vec{x} \\ \myvec{0 & -1}\vec{x}}
\end{align}
$\therefore$ The Lagrangian matrix is given by:-
\begin{align}
  \small{\myvec{0 & 0 & -1 & -1 & 1.5 & -1 & 0 \\ 0 & 0 & -2 & -0.5 & 2 & 0 & -1 \\ -1 & -2 & 0 & 0 & 0 & 0 & 0 \\ -1 & -0.5 & 0 & 0 & 0 & 0 & 0 \\ 1.5 & 2 & 0 & 0 & 0 & 0 & 0 \\ -1 & 0 & 0 & 0 & 0 & 0 & 0 \\ 0 & -1 & 0 & 0 & 0 & 0 & 0 }\myvec{\vec{x} \\ \boldsymbol{\lambda} }}= \small{\myvec{-3 \\ -3.5 \\ -240 \\ -90 \\ 310 \\ 0 \\0 }}
\end{align}
\item Considering $\lambda_1,\lambda_2$ as only active multiplier,
\begin{align}
    \myvec{0 & 0 & -1 & -1  \\ 0 & 0 & -2 & -0.5 \\ -1 & -2 & 0 & 0 \\-1 & -0.5 & 0 & 0}\myvec{\vec{x}\\ \boldsymbol{\lambda}} &= \myvec{-3 \\ -3.5 \\ -240 \\ -90}
\end{align}
\begin{align}
 \implies   \myvec{\vec{x} \\ \boldsymbol{\lambda}} &=  \myvec{0 & 0 & -1 & -1  \\ 0 & 0 & -2 & -0.5 \\ -1 & -2 & 0 & 0 \\-1 & -0.5 & 0 & 0}^{-1}\myvec{-3 \\ -3.5 \\ -240 \\ -90}
    \\
    \implies   \myvec{\vec{x} \\ \boldsymbol{\lambda}} &= \myvec{0 & 0 & \frac{1}{3} & \frac{4}{3} \\ 0 & 0 & \frac{-2}{3} & \frac{2}{3} \\ \frac{1}{3} & \frac{-2}{3} & 0 & 0 \\ \frac{-4}{3} & \frac{2}{3} & 0 & 0}\myvec{-3 \\ -3.5 \\ -240 \\ -90}
    \\
    \implies \myvec{\vec{x} \\ \boldsymbol{\lambda}} &= \myvec{40 \\100 \\ \frac{4}{3} \\ \frac{5}{3} }
\end{align}
$\because \boldsymbol{\lambda}=\myvec{\frac{4}{3} \\ \frac{5}{3}} \succ \vec{0} $
\\
\item The Optimal solution is given by:
\begin{align}
    \vec{x} &= \myvec{40\\100} \\
    Z &= \myvec{3&3.5}\vec{x} \\
   Z &= \myvec{3&3.5}\myvec{40 \\ 100} \\
    Z&= 470 \text{ units}
\end{align}
\item So, we get
\\
 Bags of brand \textbf{P} as \boxed{x=40} $\And$
 \\
 Bags of brand \textbf{Q} as  \boxed{y=100} so as to minimise the amount of nitrogen added.
\item The minimum amount of nitrogen required is \boxed{Z=470 \text{ units}} .

\numberwithin{figure}{section}
\begin{figure}[!ht]
\centering
\includegraphics[width=\columnwidth]{Graphical_Solution_2.30.png}
\caption{Graphical Solution}
\end{figure}

\end{itemize}
\end{document}
